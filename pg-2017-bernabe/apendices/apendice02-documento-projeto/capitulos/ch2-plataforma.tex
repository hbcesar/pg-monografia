\chapter{ Plataforma de Desenvolvimento}
\label{sec-plataforma}



%=======================================================================================================
%			Tabela de Plataforma de Desenvolvimento e Tecnologias Utilizadas
%=======================================================================================================

Na Tabela~\ref{tabela-plataforma} são listadas as tecnologias utilizadas no desenvolvimento da ferramenta, bem como o propósito de sua utilização.

\begin{table}[h]
	\centering	
	\vspace{0.5cm}
	\footnotesize
	\begin{tabular}{|p{1.6cm}|c|p{5cm}|p{6.5cm}|}  \hline \rowcolor[rgb]{0.8,0.8,0.8}
	
 		Tecnologia & Versão & Descrição & Propósito \\\hline 
 		                             
		JavaEE & 7 & Conjunto de especificação de APIs e tecnologias, que são implementadas por programas servidores de aplicação. & Reduzir a complexidade do desenvolvimento, implantação e gerenciamento de aplicações, de modo que o desenvolvedor não se preocupe demasiadamente com segurança, escalabilidade e desempenho. \\ \hline
	
		Java & 8 & Linguagem de programação orientada a objetos e independente de plataforma. & Desenvolvimento de aplicativos em linguagem de programação orientada a objetos e independente de plataforma. \\\hline
		
		JSF & 2.2 & Framework web baseado em Java que tem como objetivo simplificar o desenvolvimento de interfaces de sistemas para a web. & Melhorar a produtividade, permitindo a construção de interfaces para web usando um conjunto de componentes pré-construídos, ao invés de criar interfaces inteiramente do zero.  \\\hline  
		 
		EJB & 3.2 & Componente da  plataforma JEE  que roda em um container de um servidor de aplicação.   &  Fornecer um desenvolvimento rápido e simplificado de aplicações Java, com base em componentes distribuídos, transacionais, seguros e portáveis.  \\\hline
		 
		JPA &  2.1  &   API para persistência de dados por meio de mapeamento objeto-relacional.  & Eliminar muito do trabalho com queries SQL e facilitar a manuntenção visto que menos linhas de código são necessárias.\\\hline
		
		CDI & 1.1  & API para injeção de dependências.   & Integração das diferentes camadas da arquitetura e serviços de transação. \\\hline
		
		JAAS & & Serviço de Autenticação e Autorização do Java & Controlar o acesso aos recursos do sistema    \\\hline
		
		AdminLTE & 2.3.0 & Template Bootstrap 3 resposivo &  Utilizar um tamplate responsivo open source que seja facilmente personalizado, engloba scripts	JS e folhas de estilos CSS para prover um layout responsivo além de muitos outros plugins.  \\\hline
		
		Facelets & 2.0 &  Sistema de template Web de código aberto.  & Reusar estrutura comum às paginas e facilitar futura manutenção do padrão visual do sistema.  \\\hline
		 
		  PrimeFaces & 5.1 &  Conjunto de componentes JSF open source com várias extensões.  & Reutilizar componentes avançados de interface gráfica.  \\\hline
		  
		MySQL Server &  5.6.23  & Sistema Gerenciador de Banco de Dados Relacional gratuito.    &  Persistência dos dados manipulados pela ferramenta.  \\\hline
		
		WildFly & 9.0.2 &  Servidor de Aplicações para Java EE. &  Prover acesso a aplicações web por meio do protocolo HTTP (HyperText Transfer Protocol). \\\hline
		
		
		
		
		% Hibernate \vitor{Não precisa. Você vai usar JPA e aceitar a implementação que vier no servidor de aplicações, certo? Não precisa mencionar o Hibernate.} & 4.3  &  Framework de mapeamento objeto-relacional gratuito que implementa a JPA.  &  Fazer mapeamento objeto-relacional.  \\\hline
		                              
	\end{tabular}
	\caption{Plataforma de Desenvolvimento e Tecnologias Utilizadas}	
	\label{tabela-plataforma}
\end{table}






%=======================================================================================================
%			Tabela de Softwares de Apoio ao Desenvolvimento do Projeto
%=======================================================================================================

\newpage
Na Tabela~\ref{tabela-software} vemos os softwares que apoiaram o desenvolvimento de documentos e também do código fonte.

\begin{table}[h]
	\centering	
	\vspace{0.5cm}
	\begin{tabular}{|p{3cm}|c|p{5cm}|p{6cm}|}  \hline \rowcolor[rgb]{0.8,0.8,0.8}
	
 		Tecnologia & Versão & Descrição & Propósito \\\hline 
 		 
		Eclipse Java EE IDE for Web Developers  & 4.5.1 & Ambiente de desenvolvimento (IDE) para a linguagem Java.  &	Facilitar a atividade de implementação de software.	  \\\hline 
 		                            
		Astah Community & 6.9.0 & Ferramenta para modelagem em UML & Ferramenta de modelagem UML (Unified Modeling Language). \\\hline
			              
		Apache Maven    & 3.2.5 & Ferramenta de gerência de projeto baseada em project object model (POM).  &  Simplificar o download das dependências do projeto.  \\\hline 
		
		Texmaker   & 4.1-1 &  Editor de LaTeX.  &  Desenvolver os Documentos necessários para o sistema.  \\\hline       
			              
	\end{tabular}
	\caption{Softwares de Apoio ao Desenvolvimento do Projeto}	
	\label{tabela-software}
\end{table}

