
\chapter{Descrição do Minimundo}
\label{sec-minimundo}

O DI/Ufes deseja um sistema de informação para acompanhar seus alunos egressos dos cursos de graduação (Ciência da Computação e Engenharia de Computação) e de pós-graduação (Mestrado em Informática e Doutorado em Ciência da Computação). 

Para poder acessar o sistema, os egressos terão um pré-cadastro realizado por um administrador do sistema. Somente poderão ser pré-cadastrados ex-alunos que tenham se formado em algum curso oferecido pelo DI/Ufes. Para efetuar o pré-cadastro o administrador buscará os dados do egresso junto à Ufes, a saber: nome, data de nascimento, sexo, e-mail, identidade, CPF, naturalidade e nacionalidade. Também serão informados o curso em que o egresso se formou, o número de sua matrícula, o ano de ingresso e o ano de término. 

Assim que o pré-cadastro for realizado, o sistema deverá enviar um e-mail ao egresso com um link que o leva diretamente para uma página onde pode definir sua senha. Para aumentar a segurança, esta página solicita o CPF ou a matrícula do egresso para efetivar a definição de senha. Caso o egresso perca este e-mail, poderá receber outro, devendo para isso entrar no site e informar o seu CPF/matrícula, o sistema reconhecendo o egresso, enviará o e-mail.
 
Assim que for criada a senha, o sistema levará o egresso a uma página onde ele preencherá um formulário com os seguintes campos: faixa salarial, área de atuação, se atua na área em que se formou, nível de escolaridade e se reside no ES. Para cada nível de escolaridade deve dizer o título obtido, o ano, a instituição, o estado e o país.

O tempo médio exigido para o preenchimento deste formulário deve ser inferior a 5 minutos. A cada 2 anos o sistema deverá enviar um e-mail para que o usuário atualize esses dados, sendo armazenado o histórico dos mesmos.

Os egressos escolherão a sua área de atuação dentre as seguintes opções: empreendedor; funcionário público; funcionário privado; professor; ou pesquisador. E informarão se atuam em Informática, área afim ou área não correlata. Será perguntado se a formação acadêmica adquirida no curso da Ufes contribuiu para a sua atividade atual.

Os egressos escolherão a faixa salarial, dividida da seguinte forma: até 3 salários mínimos; de 3 a 5 salários mínimos; de 5 a 10 salários mínimos;  de 10 a 15 salários mínimos; de 15 a 20 salários mínimos; e acima de 20 salários mínimos. Poderão também optar por assuntos de interesse para recebimento de e-mail. A princípio os assuntos serão: Redes de Computadores e Sistemas Distribuídos; Computação de Alto Desempenho; Inteligência Computacional; Sistema de Informação; e Otimização. 

Egressos poderão postar depoimentos sobre o curso que realizaram. Esses depoimentos ficarão acessíveis a todos que acessarem o site, depois de serem avaliados e liberados pelo coordenador do curso a fim de evitar críticas gratuitas depreciativas. O egresso poderá optar por aparecer seu nome no depoimento ou se ele quer que fique anônimo. De um depoimento deseja-se saber a data de envio, sobre qual curso, o autor e o conteúdo. 

Assim como no caso dos depoimentos, os egressos também poderão mandar comentários ou sugestões sobre o curso que realizaram. Estes serão enviadas para o coordenador do curso para que possa respondê-los e também auxiliar em melhorias a serem feitas nos cursos. 

Administradores do sistema poderão cadastrar seminários, informando o assunto, o título, a data e horário, o local e o palestrante. Caso não tenha palestrante ainda, o administrador terá a opção de enviar um e-mail aos egressos convidando-os a serem o palestrante. Caso alguém responda ao chamado (por e-mail, externo ao sistema), o administrador terminaria o cadastro do seminário. Assim que a palestra estiver confirmada, o sistema enviará um e-mail para todos os egressos que tenham interesse pelo assunto, convidando-os para participarem. Os egressos também teriam a opção de sugerir um assunto em que tenham interesse em ser o palestrante. Neste caso o administrador confirmaria com ele e cadastraria o seminário no sistema.


\section{Relatórios}
\label{sec-minimundo:relatorio}

\noindent No site, ficarão disponíveis para consulta relatórios sobre dados estatísticos. Estes dados serão mostrados na forma de gráficos, assim os usuários poderão escolher um curso e optar pelos seguintes gráficos: 

\begin{itemize}
	
  	\item \textbf{Faixa Salarial:} mostra a porcentagem de egresso em cada faixa salarial.
  	
  	\item \textbf{Área de Atuação:} mostra a porcentagem de egresso em cada área: (Empreendedor), (Func. Público), (Func. Privado), (Professor) e (Pesquisador).
  	
  	\item \textbf{Atuação do Egresso:} mostra a porcentagem de egressos que atuam na área da informática, a porcentagem dos que atuam em áreas afins e a porcentagem dos que atuam em áreas não correlatas. 
  	
  	\item \textbf{Escolaridade:} mostra a porcentagem de egressos em cada nível de escolaridade. 
  	
  	\item \textbf{Reside no ES:} mostra a porcentagem de egressos que moram no Estado.
  	
  	\item \textbf{Sexo:} mostra a porcentagem de egressos do sexo masculino e feminino.
  	
  	
\end{itemize}
   
  Os usuários também poderão consultar todos os egressos, que serão mostrados na forma de lista.