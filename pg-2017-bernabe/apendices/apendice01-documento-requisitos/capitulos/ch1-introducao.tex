\chapter{Introdução}
\label{sec-intro}


Este documento apresenta os requisitos de usuário e a análise dos requisitos do sistema \imprimirtitulo. A atividade de análise de requisitos foi conduzida aplicando-se técnicas de modelagem de casos de uso, modelagem de classes e modelagem de comportamento dinâmico do sistema. Os modelos apresentados foram elaborados usando a UML.


Este documento está organizado da seguinte forma: a seção~\ref{sec-proposito} contém uma descrição do propósito do sistema; a seção \ref{sec-minimundo} apresenta uma descrição do minimundo apresentando o problema; a seção \ref{sec-requisitos} apresenta as listas de requisitos de usuário levantados junto ao cliente; a seção~\ref{sec-subsistemas} apresenta os subsistemas identificados, mostrando suas dependências na forma de um diagrama de pacotes; a seção~\ref{sec-caso-de-uso} apresenta o modelo de casos de uso, incluindo descrições de atores, os diagramas de casos de uso e descrições de casos de uso; a seção~\ref{sec-modelo-estrutural} apresenta o modelo conceitual estrutural do sistema, na forma de diagramas de classes; a seção~\ref{sec-modelo-dinamico} apresenta o modelo comportamental dinâmico do sistema, na forma de diagramas de estado; finalmente, a seção~\ref{sec-dicionario} apresenta o  dicionário do projeto, contendo as definições das classes identificadas.