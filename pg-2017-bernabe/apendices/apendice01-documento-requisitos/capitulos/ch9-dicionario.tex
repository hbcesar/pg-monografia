\chapter{Dicionário de Projeto}
\label{sec-dicionario}
\setcounter{table}{0}

Esta seção apresenta as definições das classes (e seus atributos), servindo como um glossário do projeto. As definições são organizadas por subsistema. Vale destacar que eventuais operações que estas classes vierem a ter não são listadas e descritas nesta fase do projeto.


\section{Classes}

%%%%%       CLASSE EGRESSO        %%%%%%%%%%%%%%%%%%%%%%%%%%%%%%%%%%%%%%%%%%
\subsection{Egresso} \label{Egresso}
\begin{table}[h!]
	\footnotesize
	\begin{tabular}{|p{2.6cm}|c|c|p{7.8cm}|}   \hline \rowcolor[rgb]{0.8,0.8,0.8}
		
		%\multicolumn{4}{|c|}{ \textbf{Egresso}} \\ \hline \rowcolor[rgb]{0.95,0.95,0.95}
 		
 		\textbf{Propriedade} & \textbf{Tipo} & \textbf{Obrigatório?} & \centerline{\textbf{Descrição}} \\\hline  
 		                            
		nome & Texto & x & Nome completo do egresso. \\\hline  	
		
		data-nascimento & Data & x & Data de nascimento do egresso. \\\hline	
 		
		sexo & Caractere & x & Sexo do egresso. \\\hline 
		
		email & Texto & x & Email do egresso. \\\hline
		
		senha & Texto & x & Senha do egresso. \\\hline
		                             
		identidade & Texto & x & Número de RG do egresso. \\\hline
		
		cpf & Texto & x & Número de CPF do egresso. \\\hline
		
		naturalidade & Texto & & Naturalidade do egresso. \\\hline
		
		nacionalidade & Texto & & Nacionalidade do egresso. \\\hline
		
	\end{tabular}	
\end{table}


%%%%%       CLASSE EGRESSO-HISTORICO        %%%%%%%%%%%%%%%%%%%%%%%%%%%%%%%%%%%%%%%%%%
\subsection{Histórico do Egresso} \label{Histórico-do-Egresso}
\begin{table}[h!]
	\footnotesize
	\begin{tabular}{|p{2.6cm}|c|c|p{7.8cm}|}   \hline \rowcolor[rgb]{0.8,0.8,0.8}
	
		%\multicolumn{4}{|c|}{ \textbf{Histórico do Egresso}} \\ \hline \rowcolor[rgb]{0.95,0.95,0.95}
		
 		\textbf{Propriedade} & \textbf{Tipo} & \textbf{Obrigatório?} & \centerline{\textbf{Descrição}} \\\hline  	
	
 		data-envio & Data & x & Data de envio dos dados pelo egresso. \\\hline
		                      
		faixa-salarial & Enumerado & x & Faixa onde se encontra a renda do egresso. \\\hline
		
		area-atuacao & Enumerado & x & Área onde o egresso atua profissionalmente. \\\hline
		
		escolaridade & Enumerado & x & Nível de escolaridade do Egresso. \\\hline
		
		atua-na-area & Booleano & x & Se o egresso atua na área em que se formou. \\\hline		
		
		reside-ES & Booleano & x & Se o egresso reside no estado do Espirito Santo. \\\hline	
			
	
	\end{tabular}	
\end{table}


%%%%%       CLASSE ESCOLARIDADE        %%%%%%%%%%%%%%%%%%%%%%%%%%%%%%%%%%%%%%%%%%
\subsection{Escolaridade} \label{Escolaridade}
\begin{table}[h!]
	\footnotesize
	\begin{tabular}{|p{2.6cm}|c|c|p{7.8cm}|}   \hline \rowcolor[rgb]{0.8,0.8,0.8}
	
		%\multicolumn{4}{|c|}{ \textbf{Escolaridade}} \\ \hline \rowcolor[rgb]{0.95,0.95,0.95}
		
 		\textbf{Propriedade} & \textbf{Tipo} & \textbf{Obrigatório?} & \centerline{\textbf{Descrição}} \\\hline  	
	
 		titulo & Enum & x & Titulo da graduação adiquirida. \\\hline
		
		estado & Texto & x & Nome do estado onde foi realizado. \\\hline   
 		                      
		país & Texto & x & Nome do país onde foi realizado. \\\hline
		
		ano & Data & x & Ano de conclusão. \\\hline
		
		instituição & Texto & x & Nome da instituição onde foi realizado. \\\hline	
		
			
		
	\end{tabular}	
\end{table}

\newpage

%%%%%       CLASSE CURSO        %%%%%%%%%%%%%%%%%%%%%%%%%%%%%%%%%%%%%%%%%%
\subsection{Curso} \label{Curso}
\begin{table}[h!]
	\footnotesize
	\begin{tabular}{|p{2.6cm}|c|c|p{7.8cm}|}   \hline \rowcolor[rgb]{0.8,0.8,0.8}
	
		%\multicolumn{4}{|c|}{ \textbf{Curso}} \\ \hline \rowcolor[rgb]{0.95,0.95,0.95}
		
 		\textbf{Propriedade} & \textbf{Tipo} & \textbf{Obrigatório?} & \centerline{\textbf{Descrição}} \\\hline 
 		                            
		nome & texto & x & Nome do curso. \\\hline 
		
		código & texto & x & Código do curso. \\\hline 
		                             
		
	\end{tabular}	
\end{table}



%%%%%       CLASSE ASSUNTO DE INTERESSE          %%%%%%%%%%%%%%%%%%%%%%%%%%%%%%%%%%%%%%%%%%
\subsection{Assunto de Interesse} \label{Assunto-de-Interesse}
\begin{table}[h!]
	\footnotesize
	\begin{tabular}{|p{2.6cm}|c|c|p{7.8cm}|}   \hline \rowcolor[rgb]{0.8,0.8,0.8}
	
		%\multicolumn{4}{|c|}{ \textbf{Assunto de Interesse }} \\ \hline \rowcolor[rgb]{0.95,0.95,0.95}
		
 		\textbf{Propriedade} & \textbf{Tipo} & \textbf{Obrigatório?} & \centerline{\textbf{Descrição}} \\\hline
 		                            
		nome & texto & x & Nome do assunto de interesse. \\\hline
		
		
		
	\end{tabular}	
\end{table}


%%%%%       CLASSE DEPOIMENTO          %%%%%%%%%%%%%%%%%%%%%%%%%%%%%%%%%%%%%%%%%%
\subsection{Depoimento} \label{Depoimento}
\begin{table}[h!]
	\footnotesize
	\begin{tabular}{|p{2.6cm}|c|c|p{7.8cm}|}   \hline \rowcolor[rgb]{0.8,0.8,0.8}
	
		%\multicolumn{4}{|c|}{ \textbf{Depoimento}} \\ \hline \rowcolor[rgb]{0.95,0.95,0.95}
		
		\textbf{Propriedade} & \textbf{Tipo} & \textbf{Obrigatório?} & \centerline{\textbf{Descrição}} \\\hline  	
 		                           
		conteudo & Texto & x & Conteúdo postado pelo egresso no depoimento. \\\hline
		  
  		data-envio & Data & x & Data de envio do depoimento. \\\hline 
  		
		anonimo & Booleano & x & Se o depoimento é anônimo. \\\hline  
		
		status & Enum & x & Status do depoimento. \\\hline 		
  		
	\end{tabular}	
\end{table}


%%%%%       CLASSE Sugestao          %%%%%%%%%%%%%%%%%%%%%%%%%%%%%%%%%%%%%%%%%%
\subsection{Sugestao} \label{Sugestao}
\begin{table}[h!]
	\footnotesize
	\begin{tabular}{|p{2.6cm}|c|c|p{7.8cm}|}   \hline \rowcolor[rgb]{0.8,0.8,0.8}
	
		%\multicolumn{4}{|c|}{ \textbf{Sugestao}} \\ \hline \rowcolor[rgb]{0.95,0.95,0.95}
		
 		\textbf{Propriedade} & \textbf{Tipo} & \textbf{Obrigatório?} & \centerline{\textbf{Descrição}} \\\hline
 		                            
		conteudo & texto & x & Conteúdo postado pelo egresso  na sugestão. \\\hline
		
		resposta & texto & {} & Resposta do coordenador para a sugestão do egresso. \\\hline
		  
  		data-envio & Data & x & Data de envio da sugestão. \\\hline 
  	
  		
	\end{tabular}	
\end{table}


%%%%%       CLASSE SEMINÁRIO          %%%%%%%%%%%%%%%%%%%%%%%%%%%%%%%%%%%%%%%%%%
\subsection{Seminário} \label{Seminário}
\begin{table}[h!]
	\footnotesize
	\begin{tabular}{|p{2.6cm}|c|c|p{7.8cm}|}   \hline \rowcolor[rgb]{0.8,0.8,0.8}
	
		%\multicolumn{4}{|c|}{ \textbf{Seminário}} \\ \hline \rowcolor[rgb]{0.95,0.95,0.95}
		
 		\textbf{Propriedade} & \textbf{Tipo} & \textbf{Obrigatório?} & \centerline{\textbf{Descrição}} \\ \hline
 		
		nome do palestrante & Texto & {} & Respónsavel por realizar o seminário. \\\hline
		  
  		data & Data/Hora & {} & Dia da realização do seminário. \\\hline 
  		
  		titulo & Texto & x & Título do seminário. \\\hline  
  		
  		local & Texto & {} & Local onde se realizará o seminário. \\\hline 
  		
  		confirmado & booleano & {} & se o seminário já foi confirmado. \\\hline 
  		
	\end{tabular}	
\end{table}

\newpage

%%%%%       CLASSE ADMINISTRADOR       %%%%%%%%%%%%%%%%%%%%%%%%%%%%%%%%%%%%%%%%%%
\subsection{Administrador} \label{Administrador}
\begin{table}[h!]
	\footnotesize
	\begin{tabular}{|p{2.6cm}|c|c|p{7.8cm}|}   \hline \rowcolor[rgb]{0.8,0.8,0.8}
	
		%\multicolumn{4}{|c|}{ \textbf{Administrador}} \\ \hline \rowcolor[rgb]{0.95,0.95,0.95}
		
 		\textbf{Propriedade} & \textbf{Tipo} & \textbf{Obrigatório?} & \centerline{\textbf{Descrição}} \\\hline  	
 		                            
		nome & texto & x & Nome completo do administrador. \\\hline 
		
		email & texto & x & Email do administrador. \\\hline 
		                             
		cpf & texto & x & Número do CPF do administrador. \\\hline 
		
		matricula & texto & x & Número da matricula do administrador na UFES. \\\hline 
		
		senha & texto & x & Senha para entrar no sistema. \\\hline 
		
	\end{tabular}	
\end{table}


%%%%%       CLASSE CURSO-REALIZADO       %%%%%%%%%%%%%%%%%%%%%%%%%%%%%%%%%%%%%%%%%%
\subsection{Curso Realizado} \label{Curso-Realizado}
\begin{table}[h!]
	\footnotesize
	\begin{tabular}{|p{2.6cm}|c|c|p{7.8cm}|}   \hline \rowcolor[rgb]{0.8,0.8,0.8}
	
		%\multicolumn{4}{|c|}{ \textbf{Curso Realizado}} \\ \hline \rowcolor[rgb]{0.95,0.95,0.95}
		
 		\textbf{Propriedade} & \textbf{Tipo} & \textbf{Obrigatório?} & \centerline{\textbf{Descrição}} \\\hline 
 		                            
		matricula & texto & x & Número da matrícula do egresso no curso realizado. \\\hline 
		
		ano de início & Data & x & Data de início do curso pelo egresso. \\\hline 
		                             
		ano de fim & Data & x & Data de conclusão do curso pelo egresso. \\\hline 
	
		
	\end{tabular}	
\end{table}


\newpage

\section{Tipos de Dados Específicos de Domínio}

\subsection{Area-Atuacao} \label{Area-Atuacao}
	
	\begin{itemize}[noitemsep]
		\item Áreas em que os egressos podem estar atuando. Tipo enumerado que pode assumir os seguintes valores:
		\begin{itemize}[noitemsep]
  			\item empreendedor
  			\item funcionário no setor público
  			\item funcionário no setor privado
  			\item professor
  			\item pesquisador
		\end{itemize}
	\end{itemize}



\subsection{Faixa-Salarial} \label{Faixa-Salarial}

	\begin{itemize}[noitemsep]
	  \item  Faixa salarial do egresso. Tipo enumerado que pode assumir os seguintes valores:
	  	\begin{itemize}[noitemsep]
	  		\item até 3 salários mínimos
	  		\item de 3 a 5 salários mínimos
	  		\item de 5 a 10 salários mínimos
	  		\item de 10 a 15 salários mínimos
	  		\item de 15 a 20 salários mínimos
	  		\item acima de 20 salários mínimos.
		\end{itemize}
	\end{itemize}



\subsection{Título de Escolaridade} \label{Escolaridade}

	\begin{itemize}[noitemsep]
  	\item Título do curso realizado pelo egresso. Tipo enumerado que pode assumir os seguintes valores:
  		\begin{itemize}[noitemsep]
  			\item Superior
  			\item Especialização
  			\item Mestrado
  			\item Doutorado
 	 		\item Pós-Doutorado
		\end{itemize}
	\end{itemize}




\subsection{Área de formação} \label{area-formacao}

	\begin{itemize}[noitemsep]
  	\item Relação da área em que o egresso se formação na Ufes com a que ele esta atuando. Tipo enumerado que pode assumir os seguintes valores:
  		\begin{itemize}[noitemsep]
  			\item Atua na Área
  			\item Atua em Área Correlatao
  			\item Atua em Área não Correlata
		\end{itemize}
	\end{itemize}

