% ======================================================================================================
% TCC - César Henrique Bernabé
% Capítulo 2 - Referencial Teórico

% 
% ======================================================================================================
\chapter{Referencial Teórico}
\label{sec-referencial}

Este capítulo apresenta os principais conceitos teóricos que fundamentaram a evolução do metamodelo de requisitos do \zanshin e do desenvolvimento da ferramenta \unagi. A seção~\ref{sec-referencial-engenharia-requisitos} aborda a Engenharia de Requisitos Orientada a Objetivos, destacando os principais conceitos dessa área que foram utilizados ao longo deste trabalho. A seção~\ref{sec-referencial-zanshin} apresenta o sistema \zanshin e os detalhes do metamodelo original do \textit{framework}. A seção~\ref{sec-referencial-unagi} apresenta as principais ferramentas que foram utilizadas durante o desenvolvimento do \unagi, como as funcionalidades de Desenvolvimento Orientado a Modelos (\textit{Model Driven Development} ou MDD) do \eclipse, o plugin \sirius, dentre outros.

% ======================================================================================================
% SEÇÃO Engenharia de Requisitos Orientada a Objetivos
% ======================================================================================================

\section{Engenharia de Requisitos Orientada a Objetivos}
\label{sec-referencial-engenharia-objetivos}

% Engenharia de Software
A Egenharia de Software é uma área da Ciência da Computação voltada ao estudo dos processos, métodos, técnicas, ferramentas e ambientes de suporte ao desenvolvimento de software, apoiando-se principalmente nas práticas e aplicações da área de Gerência de Projetos com o objetivo de promover melhor organização, produtividade e qualidade em todo o processo de desenvolvimento de um software ~\cite{falboEngSoft}.

% Engenharia de Requisitos
Dentro da área de Engenharia de Software, destaca-se uma importante subárea, a área de Egenharia de Requisitos de Software, focada no processo de elicitação de requisitos, considerados fatores determinantes no sucesso do desenvolvimento de um software ~\cite{falboEngReq}. Requisitos podem ser entendidos como a definição do que o sistema pode prover, ou também entendidos como o que o sistema é capaz de fazer para atingir um determinado objetivo ~\cite{pfleeger2004engenharia}.

% Objetivos
Devido ao fato de requisitos estarem diretamente ligados aos objetivos do sistema, destaca-se também a \textbf{Engenharia de Requisitos Orientada a Objetivos}, uma parte subárea de Engenharia de Requisitos. Objetivos são parte importante do processo Engenharia de Requisitos, seu propósito é indicar as principais necessidades que justificam a criação de um determinado sistema, demonstrando os casos em que as funcionalidades do mesmo satisfarão as necessidades elicitadas, além de dizer como o sistema deve ser construído para satisfazê-las ~\cite{ross1977structured}. Em uma descrição geral e resumida do processo de identificação de objetivos, pode-se dizer que o potencial software é analisado nos ambientes organizacional, operacional e técnico, onde são assim identificados os problemas de contexto e as oportunidades de solução desses problemas. Então, os objetivos são criados com foco na resolução dos problemas e das oportunidades identificadas. Tendo em mãos os objetivos do sistema devidamente refinados, os requisitos do sistema são então elaborados para que esses objetivos sejam devidamente atendidos. Além de apoiar no processo de modelagem de requisitos, objetivos são usados para apoiar outros propósitos como gerenciamento de conflitos e o processo de verificação ~\cite{van2001goal}. De acordo com ~\cite{van2001goal}, ~\cite{jackson1995software} e ~\cite{zave1997four}, objetivos podem ser reformulados em diferentes níveis de abstração, dependendo do tipo de necessidade que o sistema alvo deve atender, abrangendo desde interesses referentes a estratégias de negócios até conceitos técnicos de atividades, podendo assim refererir-se a requisitos funcionais e não-funcionais.

% Importancia Objetivos
A necessidade de uso de objetivos no processo de modelagem de sistemas de software vem se tornando cada vez mais clara a medida que analistas percebem que:
\begin{itemize}
	\item Objetivos provêm critérios claros de completude dos requisitos do sistema, permitindo também que requisitos desnecessários sejam descartados.
	\item Objetivos facilitam o processo de entendimento dos requisitos por parte dos \textit{stakeholders}.
	\item Melhora a legibilidade de documentos de especificação de requisitos, pois permite que engenheiros possam enxergar com mais clareza as alternativas de desenvolvimento dos requisitos do sistema. Além de facilitar o processo de gerenciamento de conflitos.
	\item Objetivos dirigem parte do processo de elicitação de requisitos, facilitando a identificação de boa parte deles.	
\end{itemize}

% Hardgoal, Softgoal, Quality Constraint e Domain Assumption
Diferentemente dos requisitos, objetivos podem precisar da cooperação entre diferentes tipos de refinamentos para que sejam atendidos de forma suficiente ~\cite{dardenne1993goal}. Em outras palavras, um objetivo diretamente relacionado ao sistema a ser criado torna-se um requisito, enquanto um objetivo sob responsabilidade de um agente do ambiente em que o software será executado torna-se uma Pressuposição de Domínio (ou \textit{Domain Assumptions}) e, nesse caso, são satisfeitos devido a uma regra de negócio ~\cite{van2001goal, van1998managing}. Objetivos funcionais podem ser classificados como objetivos rígidos (\textit{Hard Goals}) e objetivos fracos (\textit{Soft Goals}), estes não possuem critérios claros de satisfação, entretanto são úteis quando deseja-se comparar os melhores refinamentos ao objetivo estudado, enquanto aqueles são objetivos cujo critério de satisfação pode ser atendido de forma técnica ~\cite{dardenne1993goal}. Para que \textit{Soft Goals} tenham um parâmetro claro de satisfabilidade, são adicionados a eles as \textit{Quality Constraints}, critérios que operacionalizam os \textit{\textbf{Soft Goals}}. Por exemplo, um \textit{Soft Goal} ``Baixo Custo'' pode ser refinado na \textit{Quality Constraint} ``Custo deve ser menor que mil reais''. Por fim, ~\cite{jureta2008revisiting} define outro tipo de refinamento para especificar a atendibilidade de um objetivo: as tarefas ou \textit{Taks}, que são os passos a serem tomados para que um determinado objetivo seja cumprido. Em outras palavras, tarefas são definidas por funcionalidades do sistemas que, se executadas com sucesso, são consideradas satisfeitas ~\cite{souza2012requirement}.

% Refinamentos
Objetivos relacionam-se um com o outro através de refinamentos. Segundo ~\cite{dardenne1991goal, dardenne1993goal}, objetivos podem ser refinados usando grafos E/OU (\textit{AND/OR}). O critério de satisfabilidade de objetivos refinados em ``E'' ou ``OU'' segue os conceitos da lógica booleana: refinamentos do tipo ``E'' implicam que para que um objetivo seja considerado satisfeito, todos os sub-objetivos refinados a partir dele devem ser satisfeitos, enquanto refinamentos do tipo ``OU'' relacionam o objetivo principal com um conjunto de alternativas, ou seja, basta que um de seus refinamentos seja atendido para que ele também seja considerado alcançado.

% ======================================================================================================
% SUBSEÇÃO Modelos de Objetivos em Tempo de Execução
% ======================================================================================================

\subsection{Modelos de Objetivos em Tempo de Execução}
\label{sec-referencial-engenharia-objetivos-runtime}

% Sistemas adaptativos
Muitas vezes os requisitos de um \textit{software} precisam ser modificados durante o ciclo de execução do mesmo. Além disso, durante o processo de especificação as partes interessadas no sistema podem apresentar requisitos condicionais, ou seja, que assumem diferentes configurações dependendo da ocorrência de determinada situação ~\cite{souza2012requirement}. Em outras palavras, há a necessidade e interesse em sistemas que possam se automonitorar e, caso necessário, se adaptarem para que seus objetivos continuem sendo satisfeitos ~\cite{dalpiaz2013runtime}. Esse tipo de sistema geralmente é composto por duas partes principais: a primeira sendo o sistema em si, que executa uma tarefa para cumprir um objetivo desejado e a segunda sendo um sistema de monitoramento do primeiro, que envia ao primeiro sistema instruções de modificação de suas configurações para que seus objetivos continuem sendo atendidos ~\cite{souza2013awareness}. 

Modelos de sistemas adaptativos incluem requisitos autoconscientes, ou seja, requisitos definidos em relação ao sucesso, falha ou qualidade de serviço de outros requisitos ~\cite{souza2013awareness}. Assim, esses requisitos são considerados ``requisitos especiais'' já que sua operacionalização está relacionada a mudança de outros requisitos ~\cite{souza2012requirement}. Ademais, o comportamento do sistema é caracterizado por eventos que ocorrem em tempo de execução e que estão diretamente ligados a instâncias de objetivos ~\cite{dalpiaz2013runtime}. Assim, é importante observar que essa abordagem é considerada orientada a objetivos já que os requisitos mencionados são derivados do refinamento de objetivos elicitados para o sistema.

%Awreqs e Evoreqs
Requisitos autoconscientes são divididos em dois tipos principais: Requisitos de Percepção (\textit{Awareness Requirements} ou \awreqs) e Requisitos de Evolução (\textit{Evolution Requirements} ou \evoreqs). \awreqs são requisitos que referem-se ao estado de outros requisitos em tempo de execução, representando situações onde as partes interessadas desejam que o sistema se adapte ~\cite{souza2012requirement}. Além disso, indicam o quão critico um requisito pode ser ao descrever o grau de tolerância a falhas do mesmo ~\cite{souza2012requirement}. \evorqs são requisitos que modificam o espaço de comportamento do sistema, permitindo que novas alternativas de requisitos sejam usadas, baseando-se em um conjunto pré-definido de etapas de evolução para os requisitos monitorados ~\cite{souza2012requirement}. 


%TODO: tentar passar uma cola melhor nesse texto, e inserir outras fontes. Explicar melhor os requisitos de adaptacao e principalmente os awreqs.

% ======================================================================================================
% SUBSEÇÃO Exemplo de Caso de Uso
% ======================================================================================================
% TODO: exemplificar GORE usando o Meeting Scheduler.
\subsection{Exemplo de Modelagem de Caso de Uso}
\label{sec-referencial-engenharia-objetivos-exemplo}


% ======================================================================================================
% SEÇÃO Zanshin
% ======================================================================================================

\section{Zanshin}
\label{sec-referencial-engenharia-software-atividade-desenvolvimento-projeto}

%TODO: falar do zanshin, porém ainda sem apresentar o metamodelo, discutir arquitetura e feedback loop.


% ======================================================================================================
% SUBSEÇÃO DESENVOLVIMENTO ORIENTADO A MODELOS
% ======================================================================================================

%TODO: Falar de MDD de forma geral, com leve foco no Eclipse Modeling Tools, e entao introduzir o sirius.

\section{Desenvolvimento Orientado a Modelos}

\subsection{Sirius}

