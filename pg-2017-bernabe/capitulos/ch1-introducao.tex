% ==============================================================================
% TCC - César Henrique Bernabé
% Capítulo 1 - Introdução
% ==============================================================================


\chapter{Introdução}
\label{sec-intro}

O Departamento de Informática da Universidade Federal do Espírito Santo (DI/Ufes) visando estimular alunos do ensino médio pela área da informática, tem a intenção de implantar um sistema de acompanhamento de egresso, visto que o egresso tem a possibilidade de comparar os conhecimentos adquiridos durante sua vida acadêmica com o exercício de sua profissão. Baseado nisso, o egresso pode prestar importante contribuição, prestando depoimentos sobre o curso em que se graduou.


O acompanhamento dos egressos é um instrumento fundamental para conhecimento do perfil profissional dos graduados, tendo o propósito de buscar subsídios para melhorar a qualidade do ensino. Além disso, com informações sobre esse perfil seria possível gerar relatórios estatísticos que ficariam disponíveis na Internet para consulta.





\section{Objetivos}
\label{sec-intro-objetivos}

O objetivo geral deste trabalho é desenvolver um sistema Web que será utilizado para acompanhamento de egressos, utilizando os conceitos aprendidos ao longo do curso de Ciência da Computação. São objetivos específicos deste projeto:


\begin{itemize}
	
	\item Levantar os requisitos necessários para o sistema e Realizar a modelagem comportamental e estrutural. Documentar na Especificação de Requisitos do sistema. Esse objetivo irá utilizar os conceitos de Engenharia de Software e, em particular, Engenharia de Requisitos;	
	
  	%\item Levantar os requisitos necessários para o sistema e criar o Documento de Especificação de Requisitos. Em seguida, analisar os requisitos levantados de modo a produzir o Documento de Análise de Requisitos. Esse objetivo irá utilizar os conceitos de Engenharia de Software e, em particular, Engenharia de Requisitos;
  	
  	
  	\item Definir a arquitetura do sistema de forma que seja possível a itilização do método FrameWeb~\cite{vitorFrameWeb} e detalhar esta arquitetura em um Documento de Projeto. Esse objetivo relaciona-se com as disciplinas de Projeto de Sistemas e Desenvolvimento Web e Web Semântica (optativa);
  	
  	
  	\item Desenvolver o sistema de acordo com a estrutura definida no Documento do Projeto, utilizar frameworks existentes para auxiliar no desenvolvimento do sistema. Esse objetivo irá utilizar conceitos de Programação, Linguagens de Programação, Banco de Dados  e Desenvolvimento Web e Web Semântica (optativa);
  	 
  	 %\item Desenvolver o sistema de acordo com a estrutura definida no Documento do Projeto,  utilizando a linguagem de programação Java e demais tecnologias de aplicações Web, tais como JSF, JAAS, CDI, JPA, entre outros. Utilizar também frameworks já existentes para auxiliar no desenvolvimento do sistema. Esse objetivo irá utilizar conceitos de Programação, Linguagens de Programação, Banco de Dados  e Desenvolvimento Web e Web Semântica (optativa);

	\item Apresentar o trabalho desenvolvido e sugerir melhorias futuras para as limitações do sistema.

\end{itemize}







\section{Metodologia}
\label{sec-intro-metodologia}

A metodologia utilizada para desenvolver este trabalho foi composta pelas seguintes atividades:


\begin{enumerate}
	
	\item \textit{Revisão Bibliográfica}: Consultar boas práticas de Engenharia de Software e de Requisitos, uso de Banco de Dados Relacional, Programação Orientada a Objetos, Padrões de Projeto de Sistemas aplicados à linguagem de programação Java, entre outros.
	
	\item \textit{Elaboração da Documentação do Sistema:} Nesta etapa, foram definidos os documentos do sistema. Em primeiro lugar, foi elaborado o Documento de Especificação de Requisitos, apresentando uma descrição geral do minimundo do sistema, definição dos requisitos funcionais e não funcionais, além das regras de negócio. Também estão neste documento a apresentação dos subsistemas, casos de uso, modelo estrutural, modelo dinâmico e glossário do projeto. Por fim, foi elaborado o Documento do Projeto, contendo a arquitetura do software e projeto detalhado de cada um dos seus componentes, seguindo a abordagem FrameWeb.
	
	\item \textit{Estudo das Tecnologias:} Nesta etapa, foi necessário o estudo de tecnologias utilizadas para o desenvolvimento do sistema, tais como: Linguagem de Programação Java; Ambiente de Desenvolvimento Eclipse Java EE; Banco de Dados mySQL; JSF, CDI, JPA; PrimeFaces (utilizado para implementação da interface com o usuário); entre outras.
	
	\item \textit{Implementação e Testes:} Nesta etapa, o sistema foi implementado e testado. Sempre que uma nova funcionalidade era implementada, uma série de testes era realizada para encontrar e corrigir possíveis erros.
	
	\item \textit{Redação da Monografia:}
	Nesta etapa, foi realizada a escrita desta monografia. Vale ressaltar que a mesma foi escrita em \textit{LaTeX}\footnote{LaTeX -- http://www.latex-project.org/} utilizando o editor \textit{Texmaker}\footnote{Texmaker -- https://en.wikipedia.org/wiki/Texmaker} e o template \textit{abnTeX}\footnote{abnTeX -- http://www.abntex.net.br} que atende os requisitos das normas da ABNT (Associação Brasileira de Normas Técnicas) para elaboração de documentos técnicos e científicos brasileiros.
	
\end{enumerate}


%%% Início de seção. %%%
\section{Organização do Texto}
\label{sec-intro-organizacao}

Esta monografia é estruturada em cinco partes e contém, além da presente introdução, os seguintes capítulos:

\begin{itemize}
	\item \textbf{Capítulo \ref{sec-referencial} --} Referencial Teórico: apresenta uma revisão da literatura acerca de temas relevantes ao contexto deste trabalho, a saber: Engenharia de Software, FrameWeb e desenvolvimento Web ;
	
	\item \textbf{Capítulo \ref{sec-requisitos} --} Especificação de Requisitos: apresenta a especificação de requisitos do sistema, descrevendo o minimundo e exibindo os seus diagramas de classes e casos de uso;
	
	\item \textbf{Capítulo \ref{sec-projeto} --} Projeto Arquitetural e Implementação: apresenta a arquitetura do sistema, assim como as partes principais de sua implementação, além das principais telas do sistema;
	
	\item \textbf{Capítulo \ref{sec-conclusoes} --} Considerações Finais: apresenta as conclusões do trabalho, dificuldades encontradas, limitações e propostas de trabalhos futuros.
\end{itemize}









