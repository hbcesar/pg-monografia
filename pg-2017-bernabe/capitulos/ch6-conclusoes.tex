% ==============================================================================
% TCC - César Henrique Bernabé
% Capítulo 3 - Considerações Finais
% ==============================================================================
\chapter{Considerações Finais}
\label{sec-conclusoes}

Este capítulo apresenta as conclusões do trabalho realizado, mostrando suas contribuições na Seção~\ref{sec-consideracoes-finais-conclusoes}. Por fim, na Subseção~\ref{sec-consideracoes-finais-limitacoes-perspectivas} são apresentadas suas limitações e perspectivas de trabalhos futuros.

\section{Conclusões}
\label{sec-consideracoes-finais-conclusoes}

A proposta de um novo metamodelo para o \zanshin tem como objetivo demonstrar a importância do uso de modelos bem definidos, pois acredita-se que linguagens de modelagem foram criadas para diminuir ao máximo a ambiguidade na possíveis interpretações de comunicação de ideias. Sendo assim, o uso de metamodelos concisos e reforçados por meio do uso de definições precisas é o passo fundamental para eliminar possíveis erros de modelagem e interpretação desses modelos.

A ferramenta \unagi pode ser dividida em duas partes: o ambiente de modelagem e o módulo de conversão. A primeira parte adota a mesma sintaxe para especificação de modelos usada na literatura disponível sobre \zanshin, portanto espera-se que usuários já familiarizados com o \framework sintam-se confortáveis ao usar o modelador. Entretanto, espera-se que novos usuários também sintam-se à vontade ao usar o \unagi devido a intuitividade da sintaxe adotada, bem como a simplicidade da interface gráfica do editor. Também espera-se que o módulo conversor impulsione o uso do \zanshin ao prover uma forma simples e fácil para criação dos arquivos de especificações dos modelos específicos. 

O uso de sistemas adaptativos aparece como solução para a atual conjuntura enfrentada pela Engenharia de Software para o desenvolvimento de sistemas complexos e distribuídos. Tanto o \zanshin quanto o \unagi têm como essência o apoio ao desenvolvimento de sistemas desse tipo, pois acredita-se que esse campo de pesquisa tornar-se-á cada vez mais promissor.


\section{Limitações e Perspectivas Futuras}
\label{sec-consideracoes-finais-limitacoes-perspectivas}
Atualmente, algumas checagens de consistência e, principalmente, opções que permitam uma personalização mais refinada da sintaxe esbarram em limitações da tecnolgia do plugin \sirius, como por exemplo opção de auto organização eficiente do modelo (atualmente essa função existe, mas não funciona corretamente). Entretanto, novas funcionalidades e melhorias vêm sendo adicionadas ao plugin constantemente e acredita-se que, em breve, muitas das atuais limitações não serão mais um problema.
Em um futuro próximo, espera-se que o módulo conversor consiga importar os arquivos gerados automaticamente para o \zanshin, possibilitando assim que o usuário possa criar e simular os processos do modelo específico com ainda mais facilidade e praticidade. Assim como pretende-se trabalhar continuamente na melhoria da ferramenta e na criação de regras de análise sintática, para que os modelos produzidos sejam de qualidade cada vez maior.